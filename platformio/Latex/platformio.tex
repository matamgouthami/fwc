\def\mytitle{IDE ASSIGMNMENT}
\def\myauthor{MATAM GOUTHAMI}
\def\mymodule{Future Wireless Communications (FWC)}
\documentclass[journal,12pt,twocolumn]{IEEEtran}

\usepackage{setspace}
\usepackage{gensymb}
\usepackage{xcolor}
\usepackage{caption}
\usepackage[hyphens,spaces,obeyspaces]{url}
\usepackage[cmex10]{amsmath}
\usepackage{mathtools}
\singlespacing
\usepackage{amsthm}
\usepackage{mathrsfs}
\usepackage{txfonts}
\usepackage{stfloats}
\usepackage{cite}
\usepackage{cases}
\usepackage{subfig}
\usepackage{longtable}
\usepackage{multirow}
\twocolumn


\usepackage{graphicx}
\graphicspath{{./images/}}
\usepackage[colorlinks,linkcolor={black},citecolor={blue!80!black},urlcolor={blue!80!black}]{hyperref}
\usepackage[parfill]{parskip}
\usepackage{lmodern}
\usepackage{tikz}
\usepackage{circuitikz}
\usepackage{karnaugh-map}
\usepackage{pgf}
\usepackage[hyphenbreaks]{breakurl}

\usepackage{tabularx}
\usetikzlibrary{calc}

\renewcommand*\familydefault{\sfdefault}
\usepackage{watermark}
\usepackage{lipsum}
\usepackage{xcolor}
\usepackage{listings}
\usepackage{float}
\usepackage{titlesec}
\DeclareMathOperator*{\Res}{Res}
\renewcommand\thesection{\arabic{section}}
\renewcommand\thesubsection{\thesection.\arabic{subsection}}
\renewcommand\thesubsubsection{\thesubsection.\arabic{subsubsection}}

\renewcommand\thesectiondis{\arabic{section}}
\renewcommand\thesubsectiondis{\thesectiondis.\arabic{subsection}}
\renewcommand\thesubsubsectiondis{\thesubsectiondis.\arabic{subsubsection}}
\titlespacing{\subsection}{1pt}{\parskip}{3pt}
\titlespacing{\subsubsection}{0pt}{\parskip}{-\parskip}
\titlespacing{\paragraph}{0pt}{\parskip}{\parskip}
\newcommand{\figuremacro}[5]{
    \begin{figure}[#1]
        \centering
        \includegraphics[width=#5\columnwidth]{#2}
        \caption[#3]{\textbf{#3}#4}
        \label{fig:#2}
    \end{figure}
}

\lstset{
frame=single, 
breaklines=true,
columns=fullflexible
}

%\thiswatermark{\centering \put(400,-128.0){\includegraphics[scale=0.3]{logo}} }
\title{\mytitle}
\author{\myauthor\hspace{1em}\\\contact\\IITH\hspace{0.5em}-\hspace{0.6em}\mymodule}

\def\inputGnumericTable{}                                 %%
\lstset{
%language=C,
frame=single, 
breaklines=true,
columns=fullflexible
}
 \begin{document}
%

\theoremstyle{definition}
\newtheorem{theorem}{Theorem}[section]
\newtheorem{problem}{Problem}
\newtheorem{proposition}{Proposition}[section]
\newtheorem{lemma}{Lemma}[section]
\newtheorem{corollary}[theorem]{Corollary}
\newtheorem{example}{Example}[section]
\newtheorem{definition}{Definition}[section]
%\newtheorem{algorithm}{Algorithm}[section]
%\newtheorem{cor}{Corollary}
\newcommand{\BEQA}{\begin{eqnarray}}
\newcommand{\EEQA}{\end{eqnarray}}
\newcommand{\define}{\stackrel{\triangle}{=}}
\bibliographystyle{IEEEtran}

\vspace{3cm}
\maketitle
\tableofcontents
  \section{Question}
    Consider the minterm list form of a Boolean function $F$ given below.\\
  \( F(P,Q,R,S) = \Sigma(m(0,2,5,7,9,11) + d(3,8,10,12,14)) \).\\
  Here,$m$ denotes a minterm and $d$ denotes a don't care term.The number of essential prime implicants of the Function $F$ is
     \section{Components}
\begin{tabularx}{0.4\textwidth} { 
  | >{\centering\arraybackslash}X 
  | >{\centering\arraybackslash}X 
  | >{\centering\arraybackslash}X
  | >{\centering\arraybackslash}X |}
\hline
\textbf{Component}& \textbf{Quantity} & \textbf{Values}\\
\hline
Arduino & 1 & UNO \\  
\hline
JumperWires & 2 & M-M \\ 
\hline
Resistor & 1 & 220\\
\hline
LED &  1 & 1\\
\hline
\end{tabularx}
     \section{K-Maps}

\begin{karnaugh-map}[4][4][1][$Y_1Y_0$][$X_1X_0$]
\minterms{0,2,5,7,9,11}
\maxterms{1,4,6,13,15}
\indeterminants{3,8,10,12,14}
\implicantedge{0}{0}{2}{2}
\implicant{5}{7}
\implicant{8}{10}




\end{karnaugh-map}\\  


 \begin{figure}
\centering
\includegraphics[width=0.7\columnwidth]{figs/led1.jpg}
\caption{led}
\label{fig:led1}
\end{figure}
   
\section{Implementation}
  \begin{tabularx}{0.46\textwidth} { 
  | >{\centering\arraybackslash}X 
  | >{\centering\arraybackslash}X  | }


\hline
\textbf{Arduino PIN} & \textbf{lcd } \\ 
\hline
D2 & OUTPUT \\
\hline
GND & GND \\
\hline
\end{tabularx}

\begin{center}
    Connections
\end{center}
\section{Procedure and Output}
    1. Connect the circuit as per the above table.\\
    2. connect the led to arduino\\
\\ \begin{tabularx}{0.45\textwidth} { 
  | >{\centering\arraybackslash}X |}
  \hline
 https://github.com/matamgouthami/fwc\\/blob/main/platformio/src/gou.cpp\\
  \hline
\end{tabularx}  
\begin{figure}
\centering
\includegraphics[width=\columnwidth]{figs/connection.jpg}
\caption{Arduino connection with led}
\label{fig:connection}
\end{figure}

\bibliographystyle{ieeetr}
\end{document}
